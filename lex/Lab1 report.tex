\documentclass[a4paper]{ctexart} %CTEX报告文章格式
\usepackage[top=3cm,bottom=2cm,left=2cm,right=2cm]{geometry} % 页边距
\usepackage{indentfirst}
\usepackage{graphicx}

\setlength{\parindent}{0em}
\CTEXsetup[format={\Large\bfseries}]{section}
\CTEXsetup[name={,、},number={\chinese{section}}]{section}

\title{编译原理Lab1实验报告}
\author{Nosolution}
\date{}
\begin{document}
	\maketitle
	
	\section{目标}
	通过自己动手实现Lex程序,加深理解编译原理课程中学习到的理论知识,主要熟悉和思考正则表达式到DFA的转换过程。

	\section{项目描述}

	\subsection{项目简介}
	项目使用java语言实现,载体为maven项目。在lex包下的每一个模块(单独的文件或者子包)分别实现一部分功能。\\
	其中Main文件为项目的入口和驱动文件,Global文件定义全局需要的常量,LexSourceParser负责分析给定的lex源规则文件,ReNormalizer紧接着对读入的正则表达式进行规范化处理,ReTranslator将规范化后的正则表达式翻译为NFA,fa模块提供FA(包括NFA和DFA)的定义和交互逻辑,主要实现NFA到DFA和DFA到数据表的转换,最后由CodeGenerator生成对应的能分析符合给定规则的字符流的代码。

	\subsection{编译}

	进入pom文件所在目录,执行打包命令:

	\verb|mvn package -Dmaven.test.skip=true|

	如果成功,会在在target子目录下生成njlex-1.0.jar和njlex-1.0-jar-with-dependencies.jar文件

	\subsection{运行}

	njlex-1.0-jar-with-dependencies.jar文件已经打包好所需依赖,直接运行:

	\verb|java -jar njlex-1.0-jar-with-dependencies.jar lex <lex源规则文件路径>|

	在当前目录下生成词法分析器源文件YyLex.java。

	\subsection{使用分析器}

	编译:\verb|javac YyLex.java|\\
	运行:\verb|java YyLex <待分析文件路径>|\\
	分析结果写入名为"yy.out"的文件

	\section{基本思想和方法}

	实验实现主要采用了结构化编程的方式,那些单独的文件基本在单例的状态下解决问题(单例或者静态方法),fa包中的几个类有一定的交互,主要分为两类,FA类和FANode类。其中FANode代表一个有限状态机转换图中的一个节点,以自身的状态作为标志,知晓自己所有的转移条件和转移目标。FA类则是节点的集合,聚集了所有节点的信息并且有额外的操作,比如NFA之间可以完成concat, or, repeat的操作(后见THOMSON算法),DFA可以通过NFA构造自身,或者实现状态的最小化。可以见到类的设计并没有很好地符合面向对象的设计范式,算是时间限制下的一个不完美之处。

	\section{假设}
	假设,也就等同于本实验实现了多少功能

	\subsection{Lex源规则文件的编写方式}

	我所定义的Lex源文件规则与标准Lex的类似,主要构成:

	\begin{verbatim}
	(raws)
	definition
	%%
	(locals)
	rules
	%%
	user code
	\end{verbatim}

	考虑到java语言代码的特性,在其上做了一些修改,其中各部分解释如下:

	\begin{enumerate}
		\item raws部分: 在第一条definition之前定义
			\begin{itemize}
				\item 以空白符(空格或者制表符)开头,到本行末尾,此行会被识别为一条\verb|raw definition|
				\item   以"$\%${"开始,以"$\%$}"结束,开始结束符号单独成行,中间所有行会被识别为\verb|raw definition|
				\item \verb|raw definition|部分的所有内容会按原样输出到生成文件\verb|YyLex|的类定义之前,诸如\verb|import|或者\verb|package|之类的定义可以在此写入
			\end{itemize}
		\item definition部分:用户编写的正则表达式定义,格式\verb|<name> <definition>|,中间用空格或者制表符分隔
			\begin{itemize}
				\item \verb|name|只能由字母组成
				\item definition为\verb|name|所标志的正则表达式
				\item 在规则部分将会把所有规则中\verb|{<name>}|形式部分替换成其定义
			\end{itemize}
		\item locals部分:类似raws部分,在第一条rule之前定义,有单行和多行两种形式
			\begin{itemize}
				\item local部分的所有内容会被拷贝到\verb|yylex()|方法的头部,用户可以自行声明和初始化局部变量,希望变量名不要以"yy"开头,以免与程序中预先声明的变量冲突。
			\end{itemize}
		\item ruls部分:用户定义的正则表达式规则,格式\verb|<pattern> <action>|,中间用空格或者制表符分隔
			\begin{itemize}
				\item pattern为可匹配的正则表达式
				\item action为匹配后执行的动作,分为单行和多行
				\begin{itemize}
					\item 直接写入动作便是单行
					\item action部分为"$\%${"代表会编写多行,直到某一行除去空白符后为"$\%$}"则停止记录该pattern所指定的action
				\end{itemize}
			\end{itemize}
		\item user code部分:用户自行定义的类成员部分
			\begin{itemize}
				\item 会被拷贝到\verb|YyLex\verb|类内部,\verb|yylex()\verb|方法之外
				\item 可以包括类成员变量和类方法
			\end{itemize}
	\end{enumerate}

	\subsection{Lex源规则文件}

	类似标准Lex源文件,我也同样提供了一些可用的变量和函数,比如:

	\begin{itemize}
		\item \verb|yytext()|,代表被识别到的字符串
		\item \verb|yyleng()|,被识别字符串的长度
		\item \verb|yy_out|,输出流,类型为PrintStream
	\end{itemize}

	等,因不是此次实验的重点因此不多谈

	\subsection{正则表达式的表示范围}

	已支持的正则表达式:\\
	普通匹配,\verb^|^运算符,\verb|*|运算符,\verb|+|运算符,\verb|?|运算符,字符类(中括号表示),\verb|raw string|(引号表示)

	未支持的正则表达式,主要为所有上下文有关的正则表达式符号:\\
	\verb|^|行首符,\verb|$|行末府,\verb|/|后缀符

	\subsection{正则表达式的表示范围}
	因为jvm在编译时限制单个源文件最大为64KB,超出会编译失败,而此次实验目前在使用静态声明的数组存储DFA的信息,因而DFA的大小也就是规则部分的RE的数量和匹配能力直接影响了YyLex.java文件的大小,所以在编写Lex源规则文件的时候需要加入此方面的考虑。

	\subsection{支持的字符集}
	可以识别ASCII表中的所有可见字符,不可见字符中的\verb|\0|,\verb|\b|和\verb|\v|暂未考虑\\
	其中\verb|\b|被作为程序中的EPSILON符号使用,为了简化考虑而没有将其本身加入支持字符集

	\subsection{用户对换行符的了解}
	因为生成的分析器代码不会对读入字符有任何的额外处理,因此用户可能在跨平台的换行符识别规则上碰上困难。\\
	Windows平台的换行符号为\verb|\r\n|,类Unix平台上为\verb|\n|,我假设用户如果要识别换行,能准确地写出符合自己使用的平台的规则

	\section{FA相关描述}
	我觉得代码描述得最清楚,包括其职责和能力

	\section{重要的数据结构描述}

	\subsection{Tuple类}
	自行实现了元组,构造成列表由\verb|NFANode|使用,代表从其开始的所有转换条件和转换目标

	\subsection{Node类}
	通过FA保存\verb|Node|,\verb|Node|保存转换条件和转换目标这种间接使用来避开直接保存\verb|<source, condition, destination>|三元组这种开销很大的做法。(虽然最后实现出来好像效率也不怎么高)

	\subsection{FA类}
	\verb|NFA|和\verb|DFA|的字段如下,除了\verb|initial|,\verb|accepts|,\verb|nodes|,\verb|actIdxMap|这四个必要的字段,\verb|nodeMap|作为信息冗余,通过牺牲空间来换取时间

	\section{核心算法}
	\subsection{正则表达式后缀化}
	采用标准的中缀转后缀算法(在此前需要加入二元的连接符号),定义运算符优先级,使用栈这种数据结构来辅助完成正则表达式的后缀化。\\
	注意转义属于特殊情况,不能单独作为运算符或者运算数看待,不受后缀化影响,是后缀表达式中的特例,遇上转义符号代表其后的一个符号要转义。

	\subsection{THOMPSON算法构造NFA}
	RE后缀化之后,按顺序读取表达式,遇上普通字符则构造一个\verb|NFA|,表示只有一次转换。\\
	(nfa-construction.png)\\
	再使用栈保存已构造的\verb|NFA|,遇上运算符推出栈顶的一个或者两个NFA进行运算。\\
	(nfa-operate.png)\\
	遍历RE后即得所求NFA

	\subsection{DFA状态最小化}
	构造状态束集合,最初的元素有:所有非终态集合,根据执行动作划分的不同终态集合。\\
	注意因为此DFA负责代表所有的规则,因此执行不同动作的终态需要区分开。\\
	不断探测查看是否可以分裂当前状态束集合,直到不能再分裂为止。

	\section{用例和运行情况}
	\subsection{用例1}
	lex源规则如下

	\begin{verbatim}
	"int" 	yy_out.print("INT"); 
	"char" 	yy_out.print("CHAR");
	{W}+	yy_out.print(" ");
	\r?\n	yy_out.print("\n");
	{L}({L}|{D})*		{ 
			yy_out.print("IDENTIFIER: " + yytext());
			//indent test
			}
	\end{verbatim}

	输入如下:
	\begin{verbatim}
	int  char  
	iden char ccc
	  int  
	\end{verbatim}

	输出:
	\begin{verbatim}
	INT CHAR 
	IDENTIFIER: iden CHAR IDENTIFIER: ccc
	  INT
	\end{verbatim}

	\subsection{用例2}

	lex源规则如下

	\begin{verbatim}
	D			[0-9]
	L			[a-zA-Z_]
	H			[a-fA-F0-9]
	E			([Ee][+-]?{D}+)
	P           ([Pp][+-]?{D}+)
	W 			[ \t]
	FS			(f|F|l|L)
	%%
	"int" 	yy_out.print("INT"); 
	"char" 	yy_out.print("CHAR");
	"public" yy_out.print("PUBLIC");
	"void" yy_out.print("VOID");
	"return" yy_out.print("RETURN");
	
	"+="			{ 
			yy_out.print("ADD_ASSIGN"); 
		}
	"-="			{ 
			yy_out.print("SUB_ASSIGN"); 
		}
	"*="			{ 
			yy_out.print("MUL_ASSIGN"); 
		}
	"/="			{ 
			yy_out.print("DIV_ASSIGN"); 
		}
	"++"			{ 
			yy_out.print("INC_OP"); 
		}
	"--"			{ 
			yy_out.print("DEC_OP"); 
		}
	"->"			{ 
			yy_out.print("PTR_OP"); 
		}
	"&&"			{ 
			yy_out.print("AND_OP"); 
		}
	"||"			{ 
			yy_out.print("OR_OP"); 
		}
	"<="			{ 
			yy_out.print("LE_OP"); 
		}
	">="			{ 
			yy_out.print("GE_OP"); 
		}
	"=="			{ 
			yy_out.print("EQ_OP"); 
		}
	"!="			{ 
			yy_out.print("NE_OP"); 
		}
	";"			{ 
		yy_out.print(";"); 
		}
	"{"		{ 
			yy_out.print("{"); 
		}
	"}"		{ 
			yy_out.print("}"); 
		}
	","			{ 
			yy_out.print(","); 
		}
	":"			{ 
			yy_out.print(":"); 
		}
	"="			{ 
			yy_out.print("="); 
		}
	"("			{ 
			yy_out.print("("); 
		}
	")"			{ 
			yy_out.print(")"); 
		}
	"["		{ 
			yy_out.print("["); 
		}
	"]"		{ 
			yy_out.print("]"); 
		}
	"."			{ 
			yy_out.print("."); 
		}
	"-"			{ 
			yy_out.print("-"); 
		}
	"+"			{ 
			yy_out.print("+"); 
		}
	"*"			{ 
			yy_out.print("*"); 
		}
	"/"			{ 
			yy_out.print("/"); 
		}
	"<"			{ 
		yy_out.print("<"); 
		}
	">"			{ 
			yy_out.print(">"); 
		}
	{W}+	yy_out.print(" ");
	\r?\n	yy_out.print("\n");
	{L}({L}|{D})*		{ 
			yy_out.print("IDENTIFIER: " + yytext());
			//indent test
		}
	[1-9]{D}*{FS}?		{ 
			yy_out.print("DECIMAL_INTEGER: " + yytext()); 
	%%
	\end{verbatim}

	输入如下:
	\begin{verbatim}
	public int main() {
		int a = 1;
		int b = 2:
		a += b;
		retrun a;
	}  
	\end{verbatim}

	输出:
	\begin{verbatim}
	PUBLIC INT IDENTIFIER: main() {
		INT IDENTIFIER: a = DECIMAL_INTEGER: 1;
		INT IDENTIFIER: b = DECIMAL_INTEGER: 2:
		IDENTIFIER: a ADD_ASSIGN IDENTIFIER: b;
		IDENTIFIER: retrun IDENTIFIER: a;
	}
	\end{verbatim}

	\section{出现问题及相关解决方案}
	\begin{itemize}
		\item 正则表达式处理和DFA的优化需要十分关注细节
		\item FA中要找bug十分困难
	\end{itemize}

	解决都是靠花时间

	\section{个人感受}
	至少通过此项目我能更深刻地理解Lex中那些算法和FA的相关概念,自己编码也会涉及到许多关于具体实现上细节的考虑。\\
	除此之外,代码生成器及分析代码的框架设计也是一个很值得研究部分。可惜时间有限,不能做到进一步的优化和精细化。

	
\end{document}